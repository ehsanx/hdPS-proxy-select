%Version 2.1 April 2023
% See section 11 of the User Manual for version history
%
%%%%%%%%%%%%%%%%%%%%%%%%%%%%%%%%%%%%%%%%%%%%%%%%%%%%%%%%%%%%%%%%%%%%%%
%%                                                                 %%
%% Please do not use \input{...} to include other tex files.       %%
%% Submit your LaTeX manuscript as one .tex document.              %%
%%                                                                 %%
%% All additional figures and files should be attached             %%
%% separately and not embedded in the \TeX\ document itself.       %%
%%                                                                 %%
%%%%%%%%%%%%%%%%%%%%%%%%%%%%%%%%%%%%%%%%%%%%%%%%%%%%%%%%%%%%%%%%%%%%%

\documentclass[sn-vancouver,Numbered,lineno,pdflatex]{sn-jnl}

%%%% Standard Packages
%%<additional latex packages if required can be included here>

\usepackage{graphicx}%
\usepackage{multirow}%
\usepackage{amsmath,amssymb,amsfonts}%
\usepackage{amsthm}%
\usepackage{mathrsfs}%
\usepackage[title]{appendix}%
\usepackage{xcolor}%
\usepackage{textcomp}%
\usepackage{manyfoot}%
\usepackage{booktabs}%
\usepackage{algorithm}%
\usepackage{algorithmicx}%
\usepackage{algpseudocode}%
\usepackage{listings}%
%%%%

%%%%%=============================================================================%%%%
%%%%  Remarks: This template is provided to aid authors with the preparation
%%%%  of original research articles intended for submission to journals published
%%%%  by Springer Nature. The guidance has been prepared in partnership with
%%%%  production teams to conform to Springer Nature technical requirements.
%%%%  Editorial and presentation requirements differ among journal portfolios and
%%%%  research disciplines. You may find sections in this template are irrelevant
%%%%  to your work and are empowered to omit any such section if allowed by the
%%%%  journal you intend to submit to. The submission guidelines and policies
%%%%  of the journal take precedence. A detailed User Manual is available in the
%%%%  template package for technical guidance.
%%%%%=============================================================================%%%%

\usepackage{natbib}
\usepackage{hyperref}
\usepackage[utf8]{inputenc}
\usepackage{capt-of}
\usepackage{booktabs}
\usepackage{amssymb}
\usepackage{threeparttable}
\usepackage{float}
%\floatplacement{figure}{H}
%\floatplacement{table}{H}
\usepackage{lipsum,caption}
\geometry{
  a4paper,
  left=1in,
  right=1in,
  top=1in,
  bottom=1in,
  includeheadfoot
}
\usepackage{setspace}
%\usepackage{lineno}
%\linenumbers
\usepackage{siunitx}
\sisetup{
  mode = match,
  propagate-math-font = true,
  reset-math-version = false,
  reset-text-family = false,
  reset-text-series = false,
  reset-text-shape = false,
  text-family-to-math = true,
  text-series-to-math = true
}
\doublespacing
% Disable explicit page breaks in LaTeX
\let\clearpage\relax



\raggedbottom




% tightlist command for lists without linebreak
\providecommand{\tightlist}{%
  \setlength{\itemsep}{0pt}\setlength{\parskip}{0pt}}





\begin{document}


\title[High-Dimensional Propensity Score]{Understanding the role of
different proxy selection methods in High-Dimensional Propensity Score
Analysis}

%%=============================================================%%
%% Prefix	-> \pfx{Dr}
%% GivenName	-> \fnm{Joergen W.}
%% Particle	-> \spfx{van der} -> surname prefix
%% FamilyName	-> \sur{Ploeg}
%% Suffix	-> \sfx{IV}
%% NatureName	-> \tanm{Poet Laureate} -> Title after name
%% Degrees	-> \dgr{MSc, PhD}
%% \author*[1,2]{\pfx{Dr} \fnm{Joergen W.} \spfx{van der} \sur{Ploeg} \sfx{IV} \tanm{Poet Laureate}
%%                 \dgr{MSc, PhD}}\email{iauthor@gmail.com}
%%=============================================================%%

\author*[1,2]{\pfx{Dr.} \fnm{Mohammad Ehsanul} \sur{Karim} \dgr{MSc,
PhD}}\email{\href{mailto:ehsan.karim@ubc.ca}{\nolinkurl{ehsan.karim@ubc.ca}}}

\author[3]{\fnm{Yang} \sur{Lei} }



  \affil*[1]{\orgdiv{School of Population and Public
Health}, \orgname{University of British
Columbia}, \orgaddress{\city{Vancouver}, \country{Canada}, \postcode{V6T
1Z3}, \state{BC}, \street{2206 East Mall}}}
  \affil[2]{\orgname{St.~Paul's
Hospital}, \orgaddress{\city{Vancouver}, \country{Canada}, \postcode{V6Z
1Y6}, \state{BC}, \street{588 - 1081 Burrard Street}}}
  \affil[3]{\orgdiv{Department of Statistics}, \orgname{University of
British
Columbia}, \orgaddress{\city{Vancouver}, \country{Canada}, \postcode{V6T
1Z4}, \state{BC}, \street{Room 3182 Earth Sciences Building, 2207 Main
Mall}}}

\abstract{\textbf{Purpose}: \textbf{Methods:} \textbf{Results:}
\textbf{Conclusion:}}

\keywords{Machine learning, Propensity score, Deep learning, Causal
inference}


\pacs[JEL Classification]{C18}
\pacs[MSC Classification]{92D30, 62P10}

\maketitle

\section{Background}\label{background}

\textbf{Aim}: The aim of this research is to systematically evaluate and
compare different proxy selection methods within the context of
high-dimensional propensity score (hdPS) analysis. Specifically, the
study focuses on assessing how these methods, including alternative
variable selection approaches, perform in selecting proxy variables for
confounding adjustment compared to the traditional Bross method within
the hdPS framework. We seek to determine whether these alternative
methods offer superior performance in estimating treatment effects
compared to the default Bross formula.

\section{Methods}\label{methods}

\subsection*{Data and Simulation}\label{data-and-simulation}
\addcontentsline{toc}{subsection}{Data and Simulation}

\textbf{Motivating Example}: To explore the relationship between obesity
and the risk of diabetes, we revisited this association using data from
three cycles of the National Health and Nutrition Examination Survey
(NHANES) covering the years 2013-2014, 2015-2016, and 2017-2018
\citep{karim2024high}. This analysis was informed by a thorough review
of the existing literature
\citep{saydah2014trends, liu2013association, kabadi2012joint, ostchega2012abdominal}.
To identify relevant covariates, we constructed a causal diagram based
on established causal inference principles \citep{greenland1999causal}.
The covariates included in our analysis were carefully selected and
categorized into Demographic, Behavioral, Health History,
Access-related, and Laboratory variables \citep{karim2024high}. While
most of these variables were binary or categorical, the Laboratory
variables were continuous.

\textbf{Plasmode simulation}: To rigorously assess the performance of
the methods under consideration, we employed a plasmode simulation
framework, which is particularly well-suited for reflecting real-world
data structures and complexities \citep{franklin2014plasmode}. This
approach was modeled after the analytic dataset derived from NHANES and
involved resampling from the observed covariates and exposure
information (i.e., obesity) without altering them. By mirroring key
aspects of an actual epidemiological study, this simulation framework
offers a significant advantage over traditional Monte Carlo simulations,
which often rely on idealized assumptions.

\textbf{Simulation scenarios under consideration}: Our plasmode
simulation was conducted over 500 iterations. For the base simulation
scenario, we set the prevalence of exposure (obesity) and the event rate
(diabetes) at 30\%, with a true odds ratio (OR) parameter of 1,
corresponding to a risk difference (RD) of 0. Each simulated dataset had
a sample size of 3,000 participants. The description of other scenarios
under consideration is provided in Table \ref{table:scenarios}.

\begin{table}[ht]
\centering
\caption{Overview of Plasmode Simulation Scenarios Reflecting Varying Exposure and Outcome Prevalences Based on National Health and Nutrition Examination Survey (NHANES) Data Cycles (2013-2018)}
\label{table:scenarios}
\begin{tabular}{lcccc}
  \toprule
  \textbf{Plasmode Simulation Scenario} & \textbf{Exposure} & \textbf{Outcome} & \textbf{True} & \textbf{Sample}\\
  \textbf{} & \textbf{Prevalence} & \textbf{Prevalence} & \textbf{Odds Ratio} & \textbf{Size}\\
  \midrule
  (i) Frequent Exposure and Outcome (Base) & 30\% & 30\% & 1 & 3,000 \\
  (ii) Rare Exposure and Frequent Outcome & 5\% & 30\% & 1 & 3,000 \\
  (iii) Frequent Exposure and Rare Outcome & 30\% & 5\% & 1 & 3,000 \\
  \bottomrule
\end{tabular}
\end{table}

\textbf{True Data Generating Mechanism Used in Plasmode Simulation}: The
primary goal of this study is to evaluate various variable selection
methods under realistic conditions. To achieve this, we formulated the
outcome data based on a specific model specification that incorporates
both exposure and covariates, including investigator-specified and proxy
variables. The model specification consists of three key components:

\begin{enumerate}
\def\labelenumi{\arabic{enumi}.}
\item
  \emph{Investigator-Specified Covariates}: We retained the original
  investigator-specified covariates, which were either binary or
  categorical, reflecting how real-world studies typically operate.
\item
  \emph{Transformation of Laboratory Variables}: In real-world studies,
  it is common for analysts to lack precise knowledge of the true model
  specification. To simulate this uncertainty, we transformed the
  continuous laboratory variables using complex functions such as
  logarithmic, exponential, square root, polynomial transformations, and
  interactions. This reflects the challenges analysts face in correctly
  specifying models when dealing with continuous data.
\item
  \emph{Inclusion of Proxy Variables}: Real-world studies often deal
  with unmeasured confounding, which researchers attempt to mitigate by
  adding proxy variables. However, when a large number of proxies are
  added, some may act as noise variables, contributing little to the
  analysis. To simulate this, we selected only those binary proxy
  covariates (referred to as recurrence covariates in hdPS terminology)
  that had a relative risk (RR) of less than 0.8 or greater than 1.2
  concerning the outcome. Out of 143 proxy covariates, 94 met this
  criterion and were included in calculating a simple comorbidity burden
  measure. The remaining 49 covariates were excluded from this
  calculation and considered noise. This comorbidity burden measure was
  then incorporated into our model specification for generating the
  plasmode data.
\end{enumerate}

\textbf{Performance Measures}: From this simulation, we derived several
performance metrics to evaluate the effectiveness of the methods under
consideration: (1) bias, (2) average model standard error (SE; the
average of estimated SEs obtained from a model over repeated samples),
(3) empirical SE (the standard deviation of estimated treatment effects
across repeated samples), (4) mean squared error (MSE), (5) coverage
probability of 95\% confidence intervals, (6) bias-corrected coverage,
and (7) Zip plot \citep{morris2019using, white2023check}.

\subsection*{Estimators under
consideration}\label{estimators-under-consideration}
\addcontentsline{toc}{subsection}{Estimators under consideration}

The comparison between the data generation process and the analysis
process reveals two key differences: (i) The data generation used
transformed laboratory variables, whereas the analysis was conducted
using only the original laboratory variables. (ii) The data generation
employed a simple sum of selected proxy variables (sum of 94 proxy
covariates), while the analysis included all proxy variables (143 binary
proxies), with 49 of these acting as noise variables. These differences
help us assess how the proxy variable selection methods handle model
misspecification and the presence of noise variables.

\begin{enumerate}
\def\labelenumi{\arabic{enumi}.}
\item
  \textbf{Kitchen sink model}: This is a base model for comparison,
  where no variable selection approaches were used. All
  investigator-selected features and all proxy variables were used to
  model \citep{karim2018can}.
\item
  \textbf{Bross formula}: The Bross formula is a statistical method used
  to calculate the bias introduced by not adjusting for a covariate
  \citep{bross1966spurious}. In hdPS analysis, this formula was
  originally applied to each proxy variable to measure and rank the
  potential bias if the covariate were not adjusted for. In our
  analysis, the 100 proxies with the highest bias rankings are selected
  for further modeling \citep{schneeweiss2009high, wyss2018erratum}.
\item
  \textbf{Least Absolute Shrinkage and Selection Operator (LASSO)}:
  LASSO is a variable selection technique that limits the number of
  variables by adding a penalty term to the regression model.
  Cross-validation (CV) is used in LASSO to identify variables with
  non-zero coefficients in the best model by optimizing the penalty
  value
  \citep{franklin2015regularized, schneeweiss2017variable, karim2018can}.
\item
  \textbf{Hybrid of hdPS and LASSO}: Instead of relying solely on LASSO
  for variable selection, a hybrid approach combines the Bross formula
  and LASSO. First, hdPS variables are selected using the hdPS algorithm
  (e.g., the top 100), and then LASSO is applied to further refine the
  selection \citep{karim2018can, franklin2015regularized}.
\item
  \textbf{Elasticnet}: Elastic Net is an extension of LASSO that
  includes an additional penalty term to handle multicollinearity by
  grouping correlated features and selecting the most representative
  ones \citep{karim2018can}.
\item
  \textbf{Random Forest}: The Random Forest (RF) algorithm is an
  ensemble learning method that constructs multiple decision trees to
  perform classification \citep{breiman2001random}. It calculates the
  importance of each proxy variable based on the decrease in impurity or
  Gini importance, providing a ranking of the proxies. The top 100
  variables from this ranking are manually selected for further modeling
  \citep{schneeweiss2017variable}.
\item
  \textbf{XGBoost}: XGBoost is a gradient boosting algorithm used to
  optimize machine learning models \citep{chen2016xgboost}. It builds
  decision trees that make splits based on maximum impurity reduction,
  and it assigns an importance score to each proxy variable by
  calculating the mean decrease in impurity
  \citep{xiao2024interpretable}.
\item
  \textbf{Stepwise}: Stepwise selection is a progressive feature
  selection method that can proceed in two directions---forward or
  backward---based on the maximum adjusted R-squared. We have
  implemented two versions: (a) Forward selection (FS) starts with an
  initial model (e.g., including all investigator-selected features) and
  adds proxies to the model one at a time. (b) Backward elimination (BE)
  starts with a full model (e.g., all investigator-selected features and
  all proxy variables) and removes features one at a time based on their
  contribution to the model.
\item
  \textbf{Genetic algorithm (GA)}: GA is an evolutionary algorithm
  inspired by the theory of natural selection
  \citep{holland1975adaptation}. It operates by evolving offspring from
  a population of the fittest individuals over several generations,
  evaluating and selecting the best combination of features or variables
  that maximize prediction accuracy.
\end{enumerate}

\section{Results}\label{results}

\begin{figure}[th]

{\centering \includegraphics[width=1\linewidth,]{figures/metric_comparison_Bias} 

}

\caption{Comparison of Bias Across Different Methods in hdPS Analysis}\label{fig:unnamed-chunk-1}
\end{figure}

\begin{figure}[th]

{\centering \includegraphics[width=1\linewidth,]{figures/metric_comparison_Coverage} 

}

\caption{Comparison of Coverage Probability Across Different Methods in hdPS Analysis}\label{fig:unnamed-chunk-2}
\end{figure}

\section{Real-world analysis}\label{real-world-analysis}

\emph{Here we include full data analysis (with some summary results like
exposure and outcome prevalence, and sample size) and report OR and RD.
Also mention how many proxies were chosen (add in the picture of RD and
OR; side by side for each method, ordered my magnitude of RD), and how
many were in common with hdPS (add table).}

\begin{figure}[th]

{\centering \includegraphics[width=1\linewidth,]{manuscript_files/figure-latex/unnamed-chunk-3-1} 

}

\caption{Figure presenting a comparison of Risk Differences (RD) and Odds Ratios (OR) with 95\% confidence intervals for different methods used to evaluate the association between obesity and diabetes risk. The analysis is based on data from the National Health and Nutrition Examination Survey (NHANES) for the years 2013-2018. Methods are arranged by the number of variables used in the models.}\label{fig:unnamed-chunk-3}
\end{figure}

Table \ref{tab:method-comparison} presents a pairwise comparison of the
number of proxy features shared between different variable selection
methods used in the analysis. Each cell in the table indicates the count
of common proxy variables selected by the method in the corresponding
row and column. The diagonal cells, where the row and column methods are
the same, represent the total number of proxy variables selected
exclusively by each method.

\begin{table}[htbp]
\centering
\caption{Comparison of variable overlap of selected proxies across different methods used to evaluate the association between obesity and diabetes}
\label{tab:method-comparison}
\begin{tabular}{lccccccccc}
\toprule
 & \textbf{Bross} & \textbf{Hybrid} & \textbf{LASSO} & \textbf{Elasticnet} & \textbf{GA} & \textbf{XGBoost} & \textbf{RF} & \textbf{FS} & \textbf{BE} \\
\midrule
\textbf{Bross formula} & 100 & & & & & & & & \\
\textbf{Hybrid (Bross and LASSO)} & 49 & 49 & & & & & & & \\
\textbf{LASSO} & 47 & 47 & 60 & & & & & & \\
\textbf{Elasticnet} & 54 & 48 & 60 & 69 & & & & & \\
\textbf{Genetic algorithm (GA)} & 44 & 28 & 36 & 40 & 64 & & & & \\
\textbf{XGBoost} & 38 & 24 & 28 & 30 & 25 & 48 & & & \\
\textbf{Random Forest (RF)} & 72 & 37 & 42 & 50 & 36 & 48 & 100 & & \\
\textbf{Forward selection (FS)} & 45 & 41 & 51 & 54 & 35 & 25 & 43 & 59 & \\
\textbf{Backward elimination (BE)} & 45 & 41 & 51 & 54 & 35 & 25 & 43 & 59 & 59 \\
\bottomrule
\end{tabular}
\end{table}

\textbf{Computing time}:

\emph{Report computing time for the Real-world analysis for each method.
(add ordered table)}

\section{Discussion}\label{discussion}

\textbf{Contextualizing the literature}:

\textbf{Summary of the simulation findings}:

\textbf{Data analysis findings}:

\textbf{Future Direction}:

\textbf{Conclusion}:

\section*{List of abbreviations}\label{list-of-abbreviations}
\addcontentsline{toc}{section}{List of abbreviations}

\begin{enumerate}
\def\labelenumi{\arabic{enumi}.}
\tightlist
\item
  MSE - Mean Squared Error
\item
  SE - Standard Error
\item
  PS - Propensity Score
\item
  AE - Autoencoders
\item
  DL - Deep Learning
\item
  MARS - Multivariate Adaptive Regression Splines
\item
  SMD - Standardized Mean Difference
\item
  TMLE - Targeted Maximum Likelihood Estimation
\item
  RHC - Right Heart Catheterization
\item
  SUPPORT - Study to Understand Prognoses and Preferences for Outcomes
  and Risks of Treatments
\end{enumerate}

\section*{Declarations}\label{declarations}
\addcontentsline{toc}{section}{Declarations}

\subsection*{Ethics approval and consent to
participate}\label{ethics-approval-and-consent-to-participate}
\addcontentsline{toc}{subsection}{Ethics approval and consent to
participate}

The analysis conducted on secondary and de-identified data is exempt
from research ethics approval requirements. Ethics for this study was
covered by item 7.10.3 in University of British Columbia's Policy \#89:
Research and Other Studies Involving Human Subjects 19 and Article 2.2
in of the Tri-Council Policy Statement: Ethical Conduct for Research
Involving Humans (TCPS2).

\subsection*{Consent for publication}\label{consent-for-publication}
\addcontentsline{toc}{subsection}{Consent for publication}

\subsection*{Availability of data and
materials}\label{availability-of-data-and-materials}
\addcontentsline{toc}{subsection}{Availability of data and materials}

\subsection*{Competing interests}\label{competing-interests}
\addcontentsline{toc}{subsection}{Competing interests}

Over the past three years, MEK has received consulting fees from Biogen
Inc.~for consulting unrelated to this current work. MEK was previously
supported by the Michael Smith Foundation for Health Research Scholar
award.

\subsection*{Funding}\label{funding}
\addcontentsline{toc}{subsection}{Funding}

This work was supported by MEK's Natural Sciences and Engineering
Research Council of Canada (NSERC) Discovery Grants and Discovery
Accelerator Supplements.

\subsection*{Authors' contributions}\label{authors-contributions}
\addcontentsline{toc}{subsection}{Authors' contributions}

MEK: Conceptualization, Writing -- Original Draft, Review \& Editing YL:
Formal Analysis, Review \& Editing

\subsection*{Acknowledgements}\label{acknowledgements}
\addcontentsline{toc}{subsection}{Acknowledgements}

Not applicable.

\renewcommand\refname{References}
\bibliography{mergedbibliography.bib}


\end{document}
